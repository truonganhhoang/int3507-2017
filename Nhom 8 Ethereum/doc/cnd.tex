\documentclass[12pt]{article}
	\usepackage[utf8]{vietnam}
	\usepackage{titleps}
	\usepackage{graphicx}
	\usepackage{listings}
	\usepackage{float}
	\usepackage{underscore}
	\usepackage[bookmarks=true]{hyperref}
	\usepackage{amsmath}
	\usepackage{pgfmath,pgffor}
	\usepackage{indentfirst}
	\usepackage{pdflscape}
	\usepackage{wrapfig}
	\usepackage{amsmath}
	\usepackage[export]{adjustbox}
	\usepackage[utf8]{inputenc}
	\usepackage{longtable}
	\usepackage{subcaption}
	\usepackage{kbordermatrix}
	\usepackage{amsfonts}
	\usepackage{amssymb}
	\usepackage{subfiles}
	\usepackage{xcolor}
	\usepackage{hyperref}
	\usepackage{cases}
	\usepackage{array}
	\usepackage[T1,T5]{fontenc}
	\usepackage[left=2cm,right=2cm,top=2.5cm,bottom=2.5cm]{geometry}
	\hypersetup{
	    bookmarks=true,
	    pdftitle={Báo cáo số 1},
	    pdfauthor={Nguyễn Mạnh Cường, Phạm Minh Đức, Bùi Thị Trung Thủy, Vũ Nam Tước},                     % author
	    pdfsubject={TeX and LaTeX},
	    pdfkeywords={TeX, LaTeX, graphics, images}, % list of keywords
	    colorlinks=true,       % false: boxed links; true: colored links
	    linkcolor=blue,       % color of internal links
	    citecolor=black,       % color of links to bibliography
	    filecolor=black,        % color of file links
	    urlcolor=purple,        % color of external links
	    linktoc=page            % only page is linked
	}
	\date{}
	\def\members{{"Nguyễn Mạnh Cường", "Phạm Minh Đức", "Bùi Thị Trung Thủy", "Vũ Nam Tước"}}
	\title{
	\rule{16cm}{1pt}\vskip0.5cm
	\Huge{BÁO CÁO CÁC VẤN ĐỀ HIỆN ĐẠI}\\
	\Huge{TRONG CÔNG NGHỆ THÔNG TIN}\\
	\rule{16cm}{2pt}\vskip1cm
	\vspace{0.5cm}
	ĐỀ TÀI: TÌM HIỂU VỀ ETHEREUM\\
	\vspace{2cm}
	\large \textbf{Giảng viên}: PGS.TS Trương Anh Hoàng\\
	\vspace{1cm}
	\large \textbf {Nhóm 8}\\
	\vspace{0.7cm}
	\large \textbf {Thành viên:}\\
	\foreach \i in {0,...,3} {
		\vspace{0.25cm}
		\pgfmathparse{\members[\i]}\pgfmathresult\\ }	
	}
	\setlength{\parindent}{24pt}
	\setlength{\parskip}{.5\baselineskip}
	\newpagestyle{short}
	{\sethead{Nhóm 8}{}{Tìm hiểu về Ethereum}\headrule
	  \setfoot{}{\thepage}{}}
	\newpagestyle{long}
	{\sethead{\thesection. \sectiontitle}{}{\subsectiontitle}\headrule
	  \setfoot{}{\thepage}{}}
	
	\begin{document}
		\maketitle
		\thispagestyle{empty}
		
		\newpage
		\pagestyle{short}
		\tableofcontents
		
		\newpage
		\pagestyle{long}
	\newpage
	\section{Đặt vấn đề}
	\subfile{contain/datVanDe}
	
	\newpage
	\section{Tổng quan về chuỗi khối}
		\subsection{Khái niệm chuỗi khối}
			\subfile{contain/IntroBlockChain}
			
		\subsection{Các thành phần công nghệ của chuỗi khối}
			\subfile{contain/BlockchainComponents}
				
		\subsection{Cách hoạt động của chuỗi khối}
			\subfile{contain/HowToWork}			

	\newpage	
	\section{Tổng quan về Ethereum}	
		\subsection{Giới thiệu chung}
		\subfile{contain/IntroEthereum}	
			
		\subsection{Các khái niệm cơ bản $^{[7]}$$^{[8]}$}
		\subfile{contain/EthereumComponents}
			
		\subsection{Sự khác biệt cơ bản giữa Ethereum và Bitcoin}
		\subfile{contain/EthereumVsBitcoin}	
		\newpage
		\subsection{Lộ trình của Ethereum}
		\subfile{contain/roadmap}
		
		\subsection{Ứng dụng của Ethereum $^{[10]}$}
		\subfile{contain/EthereumApp}
	
	\newpage
	\section{Các chi tiết trong Ethereum $^{[11]}$}
		\subsection{Tổng quan}
		\subfile{contain/DrivingFactor}
	
		\subsection{Mô hình chuỗi khối}
		\subfile{contain/BlockChainParadigm}
	
		\subsection{Những quy tắc}
		\subfile{contain/Conventions}
	
		\subsection{Khối, trạng thái và các giao dịch}
		\subfile{contain/BST}
		
		\subsection{Thực hiện giao dịch}
		\subfile{contain/Thgiaodich}
		
		\subsection{Tạo hợp đồng}
		\subfile{contain/taohopdong}			
		
		\subsection{Từ cây khối tới chuỗi khối}
		\subfile{contain/BTB}
		
		\subsection{Khối hoàn thiện}
		\subfile{contain/khoihoanthien}
		
		\subsection{Thực hiện hợp đồng}
		\subfile{contain/ImplementingContracts}
		
		\subsection{Hướng đi tương lai}
		\subfile{contain/FutureDirections}	

	\section{Kết luận}
	Thông qua quá trình tìm hiểu về Ethereum, nhóm đã trình bày chi tiết về các yếu tố kỹ thuật cũng như ứng dụng của Ethereum trong hiện tại cũng như tương lai. Tại thời điểm viết báo cáo này, giá trị của một đồng Bitcoin (BTC) đã vượt qua 10000 \$ và một đồng Ethereum (ETH) đã vượt qua 400 \$, điều đó cho thấy giá trị của công nghệ này lớn đến mức nào. Không chỉ có giá trị trong lĩnh vực công nghệ tài chính, nó còn cho chúng ta một phương thức khác để phát triển ứng dụng phi tập trung đầy tiềm năng. Nhóm cũng đã phát triển một ứng dụng nhỏ để thử nghiệm một cách cơ bản, từ đó bạn hoàn toàn có thể phát triển các ứng dụng trên nền tảng này tùy vào yêu cầu bài toán mà bạn gặp phải.
	
	Từ việc tìm hiểu này, nhóm đã có hiểu biết rõ hơn về Ethereum nói riêng cũng như chuỗi khối nói chung. Chính bài tập lớn đã giúp nhóm nắm vững cụ thể hơn Ethereum là gì, hiểu nó hoạt động như thế nào và ứng đụng nó thay vì nghi ngờ, lo ngại, và chỉ nhìn vào mặt tiêu cực. Tài liệu nhóm mình viết có thể còn nhiều thiếu sót nên rất mong các bạn có thể góp ý để nhóm mình hoàn thiện bài báo cáo hơn cũng như có thể mở rộng nó hơn trong tương lai.
	
	Mong tài liệu này sẽ giúp ích được cho các bạn trong việc tìm hiểu về Ethereum cũng như tận dụng được nó trong công việc của mình.
	
	\newpage
	\section{Ứng dụng bầu cử}
	\subfile{contain/voting}

	\newpage
	\section{Tài liệu tham khảo}
	\subfile{contain/reference}
\end{document}