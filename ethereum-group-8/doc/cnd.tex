\documentclass[12pt]{article}
\usepackage[utf8]{vietnam}
\usepackage{titleps}
\usepackage{graphicx}
\usepackage{listings}
\usepackage{underscore}
\usepackage[bookmarks=true]{hyperref}
\usepackage{amsmath}
\usepackage{pgfmath,pgffor}
\usepackage{indentfirst}
\usepackage{pdflscape}
\usepackage{wrapfig}
\usepackage{amsmath}
\usepackage[export]{adjustbox}
\usepackage[utf8]{inputenc}
\usepackage{longtable}
\usepackage{subcaption}
\usepackage{kbordermatrix}
\usepackage{amsfonts}
\usepackage{amssymb}

\usepackage[T1,T5]{fontenc}
\usepackage[left=2cm,right=2cm,top=2.5cm,bottom=2.5cm]{geometry}
\hypersetup{
    bookmarks=true,
    pdftitle={Báo cáo số 1},
    pdfauthor={Nguyễn Mạnh Cường (nhóm trưởng), Phạm Bảo Cường, Trần Xuân Đạt, Phạm Minh Đức, Hoàng Thanh Hằng, Nguyễn Việt Long},                     % author
    pdfsubject={TeX and LaTeX},
    pdfkeywords={TeX, LaTeX, graphics, images}, % list of keywords
    colorlinks=true,       % false: boxed links; true: colored links
    linkcolor=blue,       % color of internal links
    citecolor=black,       % color of links to bibliography
    filecolor=black,        % color of file links
    urlcolor=purple,        % color of external links
    linktoc=page            % only page is linked
}
\date{}
\def\members{{"Nguyễn Mạnh Cường", "Phạm Minh Đức", "Vũ Nam Tước", "Bùi Thị Chung Thủy"}}
\title{
\rule{16cm}{1pt}\vskip0.5cm
\Huge{BÁO CÁO CÁC VẤN ĐỀ HIỆN ĐẠI}\\
\rule{16cm}{2pt}\vskip1cm
\vspace{0.5cm}
ĐỀ TÀI: TÌM HIỀU VỀ ETHEREUM\\
\vspace{2cm}
\large Nhóm 8\\
\vspace{2cm}
\large Thành Viên:\\
\foreach \i in {0,...,3} {
	\pgfmathparse{\members[\i]}\pgfmathresult\\ }	
}
\setlength{\parindent}{24pt}
\setlength{\parskip}{.5\baselineskip}
\newpagestyle{short}
{\sethead{Nhóm 8}{}{Quản lý tiền tiết kiệm}\headrule
  \setfoot{}{\thepage}{}}
\newpagestyle{long}
{\sethead{\thesection. \sectiontitle}{}{\subsectiontitle}\headrule
  \setfoot{}{\thepage}{}}

\begin{document}
	\maketitle
	\thispagestyle{empty}
	
	\newpage
	\pagestyle{short}
	\tableofcontents
	
	\newpage
	\pagestyle{long}
	
	\section{Giới thiệu}
		\subsection{Giới thiệu chung về BlockChain}
		BlockChain có thể coi như một quyển sổ ghi chép tài chính được phân phối ngang hàng như Torrent. Không có nhà nước hay công ty nào cai quản, BlockChain được mã hóa một cách rất cầu kỳ để ngăn chặn tuyệt đối việc giả mạo thông tin.\newline
		\indent Ba thành phần công nghệ của BlockChain
		\begin{itemize}
			\item Mạng ngang hàng: Một nhóm các máy tính ví dụ như mạng BitTorrent có khả năng giao tiếp với nhau mà không phải phụ thuộc vào một người cầm quyền ở trung tâm
			\item Mật mã bất đối xứng: Một cách cho phép những máy tính này gửi các tin nhắn được mã hóa cho những người nhận đã được xác định
			\item Phép băm mật mã: Một các để sinh một "fingerprint" nhỏ, duy nhất cho bất kỳ dữ liệu nào, cho phép so sánh một cách nhanh chóng các tập dữ liệu lớn và là một cách an toàn để xác nhận rằng dữ liệu đã được thay đổi hay chưa.
		\end{itemize}
		\subsection{Ethereum}
		Ethereum (ETH) hay còn được gọi là Bitcoin 2.0 \newline
		\indent Là một nền tảng điện toán phân tán khối chuỗi, chạy trên blockchain, thông qua việc sử dụng chức năng Hợp đồng thông minh (Smart Contract) \newline
		\indent Tiền ảo Ethereum có thể thực hiện các giao dịch, hợp đồng mạng ngang hàng thông qua đơn vị tiền ảo là Ether
		\subsubsection{Lịch sử ra đời}
		Ethereum được đề xuất vào cuối năm 2013 bởi Vitalik Buterin người Nga sinh năm 1994, một cậu thanh niên chuyên nghiên cứu về lập trình tiền ảo\newline
		\indent Vốn hoá của Ethereum đạt 25 triệu USD trong đợt mở bán lần đầu năm 2014. Kể từ đó Ethereum bắt đầu phát triển Blockchain cho riêng cũng như phát triển ngôn ngữ lập trình của mình \newline
		\indent Phiên bản beta được phát hành vào tháng 7/2015 \newline
		\indent Kể từ đầu năm trở lại đây, giá Ethereum tăng hơn 2000% trong khi Bitcoin là 150%
		
		\subsubsection{Các thành phần cơ bản của Ethereum}
		\paragraph{Gas}
		\begin{itemize}
			\item Gas là chi phí nội bộ để thực hiện một giao dịch hoặc hợp đồng trong Ethereum. 
			\item Giá trị của Gas được trả bằng một lượng ether.
			\item Giá gas cho mỗi giao dịch hay hợp đồng được thiết lập để xử lý bản chất Turing Complete của Ethereum và EVM của nó (tức là mã Ethereum Virtual Machine)- đây là một trong những ý tưởng được đưa ra để hạn chế vòng lặp vô hạn.
		\end{itemize}
		\indent Ví dụ 10 Szabo, tương đương với 0.00001 Ether hay 1 Gas có thể thực hiện một dòng mã hay vài câu lệnh. Nếu không có đủ Ether trong tài khoản để hiển thị một cuộc giao dịch hay một tin nhắn thì nó được coi là không hợp lệ.
		
		\paragraph{Hợp đồng thông minh}
		\begin{itemize}
			\item Hợp đồng thông minh là một cơ chế trao đổi xác định, được kiểm soát bởi các phương tiện kỹ thuật số mà có thể giúp cho việc thực hiện giao dịch trực tiếp giữa các thực thể mà không cần tin cậy nhau
			\item Các hợp đồng này được định nghĩa bằng cách lập trình và được chạy chính xác như mong muốn mà không bị kiểm duyệt, lừa đảo hay sự can thiệp từ bên thứ ba trung gian.
			\item Trong Ethereum, các hợp đồng thông minh được coi là các kịch bản tự trị hoặc các ứng dụng phân cấp được lưu trữ trong chuỗi khối Ethereum để thực hiện sau đó bởi EVM.
			
		\end{itemize}
		\paragraph{Máy ảo Ethereum (EVM)}
		\begin{itemize}
			\item Viết tắt của cụm từ Ethereum Virtual Machine. 
			\item Là một môi trường chạy các hợp đồng thông minh Ethereum.
			\item Nó được hoàn toàn cô lập từ mạng, hệ thống tập tin và các quá trình khác của hệ thống máy chủ.
			\item Mỗi nút Ethereum trong mạng chạy một EVM và thực hiện các hướng dẫn giống nhau.
			\item Ethereum Virtual Machines đã được lập trình trong C++, Go, Haskell, Java, Python, Ruby, Rust và WebAssembly.
			
		\end{itemize}
		\subsection{Một vài ứng của Ethereum}
		\paragraph{Hiện tại}
		\begin{itemize}
			\item Hệ thống thanh toán
			\item Đầu tư vàng
			\item Gây quỹ cộng đồng
			\item Quản lý tài chính doanh nghiệp
			
		\end{itemize}
		\paragraph{Tương lai}
		\begin{itemize}
			\item Internet of Things
			
			\item Web hosting
			
			\item Thị trường tài chính, bầu cử, bất động sản,…
			
			
		\end{itemize}
	\newpage
	\section{Phân công}
	Bảng \ref{table:one} là bảng phân công công việc cho các thành viên trong nhóm.
	\newline
	\begin{table}[ht]
		\centering
		
		\begin{tabular}{| p{6cm} | p{5cm} | p{5cm} |}
			\hline
			\textbf{Công việc}  & \textbf{Thành viên} & \textbf{Hoàn thành công việc}\\
			\hline
			
		\end{tabular}
		\label{table:one}
		\caption{Bảng Phân Công}
	\end{table}
	\newpage
	\section{Các thành phần trong Ethereum}
\end{document}